\documentclass[a4paper]{article}

\usepackage{amsmath}
\usepackage{amssymb,amsfonts}
\usepackage[catalan]{babel} % Language 
\usepackage{fontspec} 
\usepackage[margin=2cm]{geometry}
\usepackage{graphicx}
\usepackage{listings}
\usepackage[dvipsnames]{xcolor}

\setlength{\parindent}{0pt}
\setlength{\parskip}{0.2cm}



\title{Problemes APA \\ Problema 5 La fàbrica de píndoles I}
\author{Lluc Bové}
\date{Q1 2016-17}

\begin{document}

\maketitle

La companyia farmacèutica \textit{Nice Pills} ha construit una cinta transportadora que porta dues \textit{classes} de píndoles (adequades per dos tipus de malalties diferents), que anomenem $C_1$ i $C_2$. Aquestes píndoles surten en dos colors: $\{yellow,white\}$, que són detectats per una càmera. La companyia fabrica píndoles en proporcions $P(C_1) = \frac{1}{3}, P(C_2) = \frac{2}{3}$. Se'ns facilita també la inforamació sobre la distribució del color per cada classe: $P(yellow|C_1) = \frac{1}{5}$, $P(white | C_1) = \frac{4}{5}$, $P(yellow| C_2) = \frac{2}{3}$, $P(white|C_2) = \frac{1}{3}$. Es demana:
\begin{enumerate}
	
	\item\textbf{Quina és la probabilitat d'error si no s'utilitza el color per classificar?}\\
		
	La probabilitat d'error si no s'utilitza el color per classificar és el que té la regla de classificació següent:
	
	$$
	R_1 =
	\begin{cases}
	  C_1 \quad \text{si } P(C_1) > P(C_2) \\
	  C_2 \quad \text{si } P(C_1) < P(C_2)
	\end{cases}
	$$	
	
	En aquest cas és triar sempre $C_2$ per tant la probabilitat d'error és $C_1$ és a dir \boxed{\frac{1}{3}}
	
	\item\textbf{ Calcular les probabilitats $P(yellow)$ i $P(white)$ i les probabilitats $P(C_1|yellow)$,$P(C_2| yellow)$,$P(C_1 | white)$ i $P(C_2 | white)$} \\
	
	Calculem les probabilitats de $yellow$ i $white$ usant la llei de probabilitat total:
	$$
	P(yellow) = P(yellow | C_1) P(C_1) + P(yellow | C_2) P(C_2) = \frac{1}{5}\times\frac{1}{3} + \frac{2}{3}\times\frac{2}{3} = \boxed{\frac{23}{45}}
	$$
	$$
	P(white) = 1 - P(yellow) = 1 - \frac{23}{45} = \boxed{\frac{22}{45}} 
	$$
	
	Usem la fórmula de Bayes per calcular la resta de probabilitats:
	
	$$
	P(C_1 | yellow) = \frac{P(C_1)P(yellow | C_1)}{P(yellow)} = \frac{\frac{1}{3}\times \frac{1}{5}}{\frac{23}{45}} = \boxed{\frac{3}{23}}
	$$
	$$
	P(C_2 | yellow) = 1  - P(C_1 | yellow) = 1 - \frac{3}{23} = \boxed{\frac{20}{23}}
	$$
	$$
	P(C_1 | white) = \frac{P(C_1)P(white | C_1)}{P(white)} = \frac{\frac{1}{3}\times \frac{4}{5}}{\frac{22}{45}} = \boxed{\frac{6}{11}}
	$$
	$$
	P(C_2 | white) = 1 - P(C_1 | white) = 1 - \frac{6}{11} = \boxed{\frac{5}{11}}
	$$
	
	\item\textbf{ Quina és la decisió òptima per pastilles \textit{yellow}? I per pastilles \textit{white}? Quins són els \textit{odds} en ambdós casos?}\\
	
	 La regla de decisió òptima depenent del color, que l'anomenem regla de Bayes, és la següent:
	$$
	R_{bayes} =
	\begin{cases}
	C_1 \quad \text{si } P(C_1 | x) > P(C_2 | x) \\
	C_2 \quad \text{si } P(C_1 | x) < P(C_2 | x)
	\end{cases}
	$$	
	On $x$ és el color.
	
	Per tant si tenim que $x = \textit{yellow}$ aleshores la millor regla de decisió és triar $C_2$ ja que $P(C_1 | yellow) < P(C_2 | yellow)$ i els odds són de:
	
	$$
	\textit{odds} = \frac{P(yellow | C_1) P(C_1)}{P(yellow | C_2) P(C_2)} = \frac{\frac{1}{5} \times \frac{1}{3}}{\frac{2}{3} \times \frac{2}{3}} = \boxed{\frac{3}{20}}
	$$
	
	Però si tenim que $x = \text{white}$ aleshores la millor regla de decisió és escollir $C_1$ ja que $P(C_1 | white) > P(C_2 | white)$ i els odds són de:
	$$
	\textit{odds} = \frac{P(white | C_1) P(C_1)}{P(white | C_2) P(C_2)} = \frac{\frac{4}{5} \times \frac{1}{3}}{\frac{1}{3} \times \frac{2}{3}} = \boxed{\frac{6}{5}}
	$$
	
	\item\textbf{Quina és la probabilitat d'error si s'utilitza el color per classificar? Per què és millor que la de l'apartat 1?}\\
	
	La probabilitat d'error és el de la regla de Bayes, per tant tenim que:
	
	$$P(Error_{R_{Bayes}} | x) = \min\{P(C_1 | x),P(C_2 | x)\}$$
	
	I la probabilitat d'error total és:
	
	$$
	E()
	$$

\end{enumerate}





\end{document}