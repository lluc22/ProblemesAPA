\documentclass[a4paper]{article}

\usepackage{amsmath}
\usepackage{amssymb,amsfonts}
\usepackage[catalan]{babel} % Language 
\usepackage{fontspec} 
\usepackage[margin=2cm]{geometry}
\usepackage{graphicx}
\usepackage[makeroom]{cancel}
\usepackage{listings}
\usepackage[dvipsnames]{xcolor}
	

\usepackage{xcolor}
\usepackage{listings}

\usepackage{xparse}
\setlength{\parindent}{0pt}
\setlength{\parskip}{0.2cm}

\usepackage{color}
\lstset{ %
	language=R,                     % the language of the code
	basicstyle=\footnotesize,       % the size of the fonts that are used for the code
	numbers=left,                   % where to put the line-numbers
	numberstyle=\tiny\color{gray},  % the style that is used for the line-numbers
	stepnumber=1,                   % the step between two line-numbers. If it's 1, each line
	% will be numbered
	numbersep=5pt,                  % how far the line-numbers are from the code
	backgroundcolor=\color{white},  % choose the background color. You must add \usepackage{color}
	showspaces=false,               % show spaces adding particular underscores
	showstringspaces=false,         % underline spaces within strings
	showtabs=false,                 % show tabs within strings adding particular underscores
	frame=single,                   % adds a frame around the code
	rulecolor=\color{black},        % if not set, the frame-color may be changed on line-breaks within not-black text (e.g. commens (green here))
	tabsize=2,                      % sets default tabsize to 2 spaces
	captionpos=b,                   % sets the caption-position to bottom
	breaklines=true,                % sets automatic line breaking
	breakatwhitespace=false,        % sets if automatic breaks should only happen at whitespace
	title=\lstname,                 % show the filename of files included with \lstinputlisting;
	% also try caption instead of title
	keywordstyle=\color{blue},      % keyword style
	commentstyle=\color{dkgreen},   % comment style
	stringstyle=\color{mauve},      % string literal style
	escapeinside={\%*}{*)},         % if you want to add a comment within your code
	morekeywords={*,...},
	          % if you want to add more keywords to the set
	alsoletter={.}        % if you want to add more keywords to the set
} 


\title{Problemes APA \\ Problema 12: Clustering de les dades artificials de Cassini}
\author{Lluc Bové}
\date{Q1 2016-17}

\begin{document}

\maketitle

Volem analitzar un problema d'agrupament amb dades en 2D usant la rutina \lstinline{mlbench.cassini}. Generem dades en 3 grups amb el codi:\\

\begin{lstlisting}[frame=none,numbers=none]
	library(mlbench)
	
	N <- 2000
	
	data.1 <- mlbench.cassini(N, relsize = c(1,1,0.25))
	
	plot(data.1)
\end{lstlisting}
 
 Veiem que les estructures externes tenen forma de plàtan i entre elles hi ha un cercle amb menys densitat de dades. El \lstinline|plot| anterior mostr la veritat de les dades (els $3$ grups generats). Si ara fem:
 
 \begin{lstlisting}[frame=none,numbers=none]
plot(x=data.1$x[,1], y=data.1$x[,2])
 \end{lstlisting}
 
 Veurem les dades en brut (el que rebrà el mètode de \textit{clustering}). Es demana:
 
 \begin{enumerate}
 	\item \textbf{ Decidiu per endavant quin mètode de \textit{clustering} hauria de treballar millor i amb quins paràmetres}
 	\item \textbf{Apliqueu k-means un cert nombre de vegades amb $k = 3$ i observeu els reultats}
 	\item \textbf{Apliqueu k-means amb una selecció de valors de k al vostre criteri (20 cops cadascun) i monitoritzeu l'índex de Calinski-Harabasz mitjà; quin k es veu millor?}
 	
 	\item \textbf{Apliqueu l'algorisme E-M amb una selecció de valors de k al vostre criteri (10 cops cadascun) i observeu els resultats. Comproveu els resultats contra les vostres expectatives (apartat 1).}
 \end{enumerate}




\end{document}